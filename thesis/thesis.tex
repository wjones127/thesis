% This is the Reed College LaTeX thesis template. Most of the work
% template. Later comments etc. by Ben Salzberg (BTS). Additional
% restructuring and APA support by Jess Youngberg (JY).
% Your comments and suggestions are more than welcome; please email
% them to cus@reed.edu
%
% See http://web.reed.edu/cis/help/latex.html for help. There are a
% great bunch of help pages there, with notes on
% getting started, bibtex, etc. Go there and read it if you're not
% already familiar with LaTeX.
%
% Any line that starts with a percent symbol is a comment.
% They won't show up in the document, and are useful for notes
% to yourself and explaining commands.
% Commenting also removes a line from the document;
% very handy for troubleshooting problems. -BTS

% As far as I know, this follows the requirements laid out in
% the 2002-2003 Senior Handbook. Ask a librarian to check the
% document before binding. -SN

%%
%% Preamble
%%
% \documentclass{<something>} must begin each LaTeX document
\documentclass[12pt,twoside]{reedthesis}
% Packages are extensions to the basic LaTeX functions. Whatever you
% want to typeset, there is probably a package out there for it.
% Chemistry (chemtex), screenplays, you name it.
% Check out CTAN to see: http://www.ctan.org/
%%
\usepackage{graphicx,latexsym}
\usepackage{amssymb,amsthm,amsmath}
\usepackage{longtable,booktabs,setspace}
\usepackage{chemarr} %% Useful for one reaction arrow, useless if you're not a chem major
\usepackage{rotating}

% Modified by CII
\usepackage[hyphens]{url}
\usepackage{hyperref}

% Added by CII (Thanks, Hadley!)
% Use ref for internal links
\renewcommand{\hyperref}[2][???]{\autoref{#1}}
\def\chapterautorefname{Chapter}
\def\sectionautorefname{Section}
\def\subsectionautorefname{Subsection}

\usepackage{caption}
\captionsetup{width=5in}

% \usepackage{times} % other fonts are available like times, bookman, charter, palatino

\title{Multilevel Models for Crowdsourced Route Ratings}
\author{Will Jones}
% The month and year that you submit your FINAL draft TO THE LIBRARY (May or December)
\date{May 2016}
\division{Mathematics and Natural Sciences}
\advisor{Andrew Bray}
%If you have two advisors for some reason, you can use the following
%\altadvisor{Your Other Advisor}
%%% Remember to use the correct department!
\department{Mathematics}
% if you're writing a thesis in an interdisciplinary major,
% uncomment the line below and change the text as appropriate.
% check the Senior Handbook if unsure.
%\thedivisionof{The Established Interdisciplinary Committee for}
% if you want the approval page to say "Approved for the Committee",
% uncomment the next line
%\approvedforthe{Committee}

% Below added by CII

\renewcommand{\contentsname}{Table of Contents}

\setlength{\parskip}{0pt}

\providecommand{\tightlist}{%
  \setlength{\itemsep}{0pt}\setlength{\parskip}{0pt}}

\Acknowledgements{
I want to thank a few people.
}

\Dedication{
You can have a dedication here if you wish.
}

\Preface{
This is an example of a thesis setup to use the reed thesis document
class.
}

\Abstract{
The preface pretty much says it all.
}


%%
%% End Preamble
%%
%

\begin{document}

      \maketitle
  
  \frontmatter % this stuff will be roman-numbered
  \pagestyle{empty} % this removes page numbers from the frontmatter

      \begin{acknowledgements}
      I want to thank a few people.
    \end{acknowledgements}
  
      \begin{preface}
      This is an example of a thesis setup to use the reed thesis document
      class.
    \end{preface}
  
  % Add table of abbreviations?

      \hypersetup{linkcolor=black}
    \setcounter{tocdepth}{2}
    \tableofcontents
  
      \listoftables
  
      \listoffigures
  
      \begin{abstract}
      The preface pretty much says it all.
    \end{abstract}
  
      \begin{dedication}
      You can have a dedication here if you wish.
    \end{dedication}
  
  \mainmatter % here the regular arabic numbering starts
  \pagestyle{fancyplain} % turns page numbering back on

  \chapter*{Introduction}\label{introduction}
  \addcontentsline{toc}{chapter}{Introduction}
  
  Knock Software's \emph{Ride Report} app combines a simple
  thumbs-up/thumbs-down rating system with GPS traces of bicycle rides to
  compile a crowdsourced data set of which routes are and are not
  stressful for urban bicyclists.
  
  The app that collects the data is simple: \emph{Ride Report}
  automatically detects when a user start riding their bike, records the
  GPS trace of the route, and then prompts the user at the end of the ride
  to give either a thumbs-up or thumbs-down rating. From this, they were
  able to create a crude ``stress map'' of Portland, OR, which simply
  shows the average ride rating of rides going through each discretized
  ride segment.
  
  The app privileges reducing barriers to response to increase sample size
  over ensuring quality and consistent responses. This presents the first
  problem: how can we analyze ratings when riders are likely rating rides
  inconsistently?
  
  At the same time, we have another challenge. We have ratings associated
  with routes, but we would like to know the effect of particular road
  segments, for both inference (what effect does this road segment have on
  the rating?) and prediction (given a route, what do we expect the rating
  to be?) purposes.
  
  \section{Accounting for Rider Rating
  Variance}\label{accounting-for-rider-rating-variance}
  
  For ratings we are interested in modeling variance between riders (as we
  might expect different rides to rate differently on average) and within
  riders (as riders may not rate the same route and conditions the same
  every time). To model this, we propose using multilevel regression, with
  random effects from each rider. This approach has been used in similar
  situations, in one case to model sexual attraction\footnote{Mackaronis,
    Strassberg, Cundiff, \& Cann (2013)}.
  
  In a multilevel model, we fit a regression where a slope of intercept
  term is a random variable whose distribution is unique to all the
  groupings. For example, if we let \(r_i\) be the rating of the \(i\)th
  ride, \(X_i\) be the ride-level variables, then we can fit a regression:
  
  \[\mathbb{P}(r_i = 1) = \text{logit}^{-1} 
  \left( \alpha_{j[i]} + \beta \cdot X_i \right) ,\] where \(\alpha_j\) is
  the contribution of the \(j\) rider:
  \[\alpha_j \sim N (\mu_\alpha, \sigma^2_j).\]
  
  We explore multilevel model further in Section 2.1 and multilevel models
  for riders in Section 4.2.
  
  \section{Addressing Road Segments as a
  Level}\label{addressing-road-segments-as-a-level}
  
  We examine multiple approaches to modeling road segments. In the first,
  we regard road segments as groups rides belong to, with the catch that
  rides can belong to multiple of these groups.
  
  \chapter{Data Sources}\label{rmd-basics}
  
  Here is a brief introduction into using \emph{R Markdown}.
  \emph{Markdown} is a simple formatting syntax for authoring HTML, PDF,
  and MS Word documents. \emph{R Markdown} provides the flexibility of
  \emph{Markdown} with the implementation of \textbf{R} input and output.
  For more details on using \emph{R Markdown} see
  \url{http://rmarkdown.rstudio.com}.
  
  Be careful with your spacing in \emph{Markdown} documents. While
  whitespace largely is ignored, it does at times give \emph{Markdown}
  signals as to how to proceed. As a habit, try to keep everything left
  aligned whenever possible, especially as you type a new paragraph. In
  other words, there is no need to indent basic text in the Rmd document
  (in fact, it might cause your text to do funny things if you do).
  
  \section{Ride Report}\label{ride-report}
  
  It's easy to create a list. It can be unordered like
  
  \begin{itemize}
  \itemsep1pt\parskip0pt\parsep0pt
  \item
    Item 1
  \item
    Item 2
  \end{itemize}
  
  or it can be ordered like
  
  \begin{enumerate}
  \def\labelenumi{\arabic{enumi}.}
  \itemsep1pt\parskip0pt\parsep0pt
  \item
    Item 1
  \item
    Item 2
  \end{enumerate}
  
  Notice that I intentionally mislabeled Item 2 as number 4.
  \emph{Markdown} automatically figures this out! You can put any numbers
  in the list and it will create the list. Check it out below.
  
  To create a sublist, just indent the values a bit (at least four spaces
  or a tab). (Here's one case where indentation is key!)
  
  \begin{enumerate}
  \def\labelenumi{\arabic{enumi}.}
  \itemsep1pt\parskip0pt\parsep0pt
  \item
    Item 1
  \item
    Item 2
  \item
    Item 3
  
    \begin{itemize}
    \itemsep1pt\parskip0pt\parsep0pt
    \item
      Item 3a
    \item
      Item 3b
    \end{itemize}
  \end{enumerate}
  
  \section{Weather Data}\label{weather-data}
  
  Make sure to add white space between lines if you'd like to start a new
  paragraph. Look at what happens below in the outputted document if you
  don't:
  
  Here is the first sentence. Here is another sentence. Here is the last
  sentence to end the paragraph. This should be a new paragraph.
  
  \emph{Now for the correct way:}
  
  Here is the first sentence. Here is another sentence. Here is the last
  sentence to end the paragraph.
  
  This should be a new paragraph.
  
  \section{Road Data}\label{road-data}
  
  When you click the \textbf{Knit} button above a document will be
  generated that includes both content as well as the output of any
  embedded \textbf{R} code chunks within the document. You can embed an
  \textbf{R} code chunk like this (\texttt{cars} is a built-in \textbf{R}
  dataset):
  
  \chapter{Data Transformation}\label{data-trans}
  
  \section{Working in Road Networks}\label{working-in-road-networks}
  
  \section{Using Nearest Neighbor Search for Map Matching
  Data}\label{using-nearest-neighbor-search-for-map-matching-data}
  
  \chapter{Methods}\label{methods}
  
  \section{Logistic Regression}\label{logistic-regression}
  
  \section{Multilevel Models}\label{multilevel-models}
  
  \chapter{Model 1: Rides and Riders}\label{model-1-rides-and-riders}
  
  \section{Choosing Ride-Level
  Parameters}\label{choosing-ride-level-parameters}
  
  \section{Adding Random Effects from
  Riders}\label{adding-random-effects-from-riders}
  
  \section{Evaluating the Ride-Level
  Models}\label{evaluating-the-ride-level-models}
  
  \chapter{Model 2: Segments as a New
  Level}\label{model-2-segments-as-a-new-level}
  
  \section{Choosing Segment-Level
  Parameters}\label{choosing-segment-level-parameters}
  
  \section{Evaluating Segment-Level
  Models}\label{evaluating-segment-level-models}
  
  \chapter{Model 3: A Spatial Model}\label{model-3-a-spatial-model}
  
  \chapter{Comparative Evaluation}\label{comparative-evaluation}
  
  \chapter*{Conclusion}\label{conclusion}
  \addcontentsline{toc}{chapter}{Conclusion}
  
  \setcounter{chapter}{4} \setcounter{section}{0}
  
  If we don't want Conclusion to have a chapter number next to it, we can
  add the \texttt{\{.unnumbered\}} attribute. This has an unintended
  consequence of the sections being labeled as 3.6 for example though
  instead of 4.1. The \LaTeX~commands immediately following the Conclusion
  declaration get things back on track.
  
  \subsubsection{More info}\label{more-info}
  
  And here's some other random info: the first paragraph after a chapter
  title or section head \emph{shouldn't be} indented, because indents are
  to tell the reader that you're starting a new paragraph. Since that's
  obvious after a chapter or section title, proper typesetting doesn't add
  an indent there.
  
  \backmatter
  
  \chapter{References}\label{references}
  
  \noindent
  
  \setlength{\parindent}{-0.20in} \setlength{\leftskip}{0.20in}
  \setlength{\parskip}{8pt}
  
  Cressie, N., \& Wikle, C. K. (2011). \emph{Statistics for
  spatio-temporal data}. John Wiley \& Sons.
  
  Gelman, A., \& Hill, J. (2006). \emph{Data analysis using regression and
  multilevel/Hierarchical models}. The Edinburgh Building, Cambridge CB2
  8RU, UK: Cambridge University Press, New York.
  
  Mackaronis, J. E., Strassberg, D. S., Cundiff, J. M., \& Cann, D. J.
  (2013). Beholder and beheld: A multilevel model of perceived sexual
  appeal. \emph{Archives of Sexual Behavior}.


  % Index?

\end{document}

